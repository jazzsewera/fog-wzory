\documentclass[12pt]{article}
\usepackage[margin=1in]{geometry} 
\usepackage{polski}
\usepackage[utf8]{inputenc}
\usepackage[polish]{babel}
\usepackage{amsmath,amsthm,amssymb,amsfonts, fancyhdr, graphicx, environ}
\usepackage{color}
\definecolor{grey}{rgb}{0.6,0.6,0.6}
\pagestyle{fancy}
\setlength{\headheight}{65pt}
\setlength{\parskip}{2em}
\newenvironment{wzor}[1]{\par{\Large $\longrightarrow$ \textit{#1}}}
    {\newline {\color{grey} \rule{\linewidth}{0.3pt}}}
\DeclareMathOperator{\der}{\operatorname{d}\!}
\newenvironment{bottompar}{\par\vspace*{\fill}}{\clearpage}
%%%%%%%%%%%%%%%%%%%%%%%%%%%%%%%%%%%%%%%%%%%%%%%%%%%%%%%%%%%%%%%%%%%%%%%%%%%%%%%%%

%%%%%%%%%%%%%%%%%%%%%%%%%%%%%%%%%%%%%%%%%%%%%
%Fill in the appropriate information below
\lhead{EiTI}  %replace with your name
\rhead{Fizyka --- Wzory}
%%%%%%%%%%%%%%%%%%%%%%%%%%%%%%%%%%%%%%%%%%%%%


%%%%%%%%%%%%%%%%%%%%%%%%%%%%%%%%%%%%%%
%Do not alter this block.
\begin{document}
%%%%%%%%%%%%%%%%%%%%%%%%%%%%%%%%%%%%%%

\begin{wzor}{Gradient}
    \begin{equation}
        \nabla f(x,y,z) = \left[
                \frac{\partial f}{\partial x},\,
                \frac{\partial f}{\partial y},\,
                \frac{\partial f}{\partial z}
        \right]
    \end{equation}
    Gradient to pole wektorowe wskazujące kierunki najszybszych wzrostów wartości
    danego pola skalarnego.
\end{wzor}

\begin{wzor}{Strumień pola}
    \begin{equation}
        \Phi_S = \int\limits_S \Vec{G} \, \der \Vec{s}
    \end{equation}
    Gdy $\Phi = 0$, linie sił biegną w sposób ciągły wewnątrz pewnej zamkniętej powierzchni,
    czyli tyle samo linii do niej wchodzi i z niej wychodzi;\\
    Gdy $\Phi > 0$, wewnątrz zamkniętej krzywej występują źródła linii sił pola;\\
    Gdy $\Phi < 0$, wewnątrz zamkniętej krzywej występują zlewy linii sił pola.
\end{wzor}

\begin{wzor}{Dywergencja}
    \begin{equation}
        \operatorname{div} \mathbf{F}(x,y,z) = \frac{\partial F_1(x,y,z)}{\partial x}
                + \frac{\partial F_2(x,y,z)}{\partial y}
                + \frac{\partial F_3(x,y,z)}{\partial z}
    \end{equation}
    Miara ,,rozbieżności'' pola wektorowego w otoczeniu danego punktu.
\end{wzor}

\begin{wzor}{Twierdzenie Stokesa}
    \begin{equation}
        \oint\limits_C \Vec{F}\,\der \Vec{l}
            = \iint\limits_{S(C)} \nabla \times \Vec{F}\,\der\Vec{a} 
    \end{equation}
    Twierdzenie umożliwia zamianę całki po krawędzi $C$ na całkę powierzchniową po powierzchni
    ograniczonej tą krawędzią $S(C)$ i na odwrót.
\end{wzor}

\begin{wzor}{Twierdzenie Ostrogradskiego - Gaussa}
    \begin{equation}
        \oint\limits_S \Vec{F} \, \der \Vec{a} = \iiint\limits_{V(S)} \nabla \Vec{F} \, \der\tau
    \end{equation}
    Twierdzenie to umożliwia zamianę całki powierzchniowej na objętościową i na odwrót.
\end{wzor}

\begin{wzor}{Rotacja}
    \begin{equation}
        \mathrm{rot}(\Vec{F}) = \nabla \times \Vec{F}
    \end{equation}
\end{wzor}

%%%%%%%%%%%%%%%%%%%%%%%%%%%%%%%%%%%%%%%%
\newpage
%%%%%%%%%%%%%%%%%%%%%%%%%%%%%%%%%%%%%%%%

\begin{wzor}{Cyrkulacja pola wektorowego}
    \begin{equation}
        \Gamma = \oint\limits_C \mathbf{G} \, \der s
            = \iint\limits_S (\nabla \times \mathbf{G}) \, \der S
    \end{equation}
    Cyrkulacja pola wektorowego po danym konturze zamkniętym $C$ jest równa strumieniowi
    rotacji tego pola wektorowego na obszarze $S = \overline{C}$ ograniczonym tym konturem.
\end{wzor}

\begin{wzor}{Potencjały pola wektorowego}
    \par \textit{Potencjał skalarny}
    \begin{equation}
        \mathbf{F} = - \nabla V
    \end{equation}
    $\mathbf{F}$ -- pole wektorowe\\
    $V$ -- potencjał skalarny pola $\mathbf{F}$
    \par \textit{Potencjał wektorowy}
    \begin{equation}
        \mathbf{F} = \nabla \times \mathbf{A}
    \end{equation}
    $\mathbf{A}$ -- potencjał wektorowy (który też jest polem wektorowym) pola $\mathbf{F}$
\end{wzor}

\begin{wzor}{Siła Coulomba}
    \begin{equation}
        F = k \frac{|q_1 q_2|}{r^2}
    \end{equation}
    Jest to siła oddziałująca między dwoma ładunkami elektrycznymi.
\end{wzor}

\begin{wzor}{Pole elektryczne}
    \begin{equation}
        \Vec{E} = \int \der \Vec{E}
    \end{equation}
    Pole elektryczne wytworzone przez ciągły rozkład ładunków, gdzie $\der\Vec{E}$ to
    natężenie pola pochodzące od pojedynczego ładunku punktowego $\der q$.
\end{wzor}

\begin{wzor}{Potencjał elektryczny}
    \begin{equation}
        V(r) = - \int\limits_P^R \Vec{E} \, \der\Vec{l}
    \end{equation}
    Potencjał jest to praca potrzebna na przemieszczenie ładunku z punktu odniesienia
    do danego punktu
\end{wzor}

%%%%%%%%%%%%%%%%%%%%%%%%%%%%%%%%%%%%%%%%
\newpage
%%%%%%%%%%%%%%%%%%%%%%%%%%%%%%%%%%%%%%%%

\begin{wzor}{Potencjał ładunku punktowego}
    \begin{equation}
        \begin{split}
            V_P &= \frac{W_{P \to \infty}}{q}\\
            W_{P \to \infty} &= E_P - E_\infty\\
            E_P &= \frac{1}{4\pi\varepsilon_0} \cdot \frac{qQ}{r}\\
            V_P &= \frac{W_{P \to \infty}}{q} = \frac{E_P}{q} - \frac{E_\infty}{q} = \frac{E_P}{q} - 0\\
            &= \frac{1}{4\pi\varepsilon_0} \cdot \frac{Q}{r}
        \end{split}
    \end{equation}
\end{wzor}

\begin{wzor}{Potencjał dla ciągłego rozkładu ładunków}
    \begin{equation}
        V(r) = \frac{1}{4\pi\varepsilon_0} \int \frac{\der q}{r}
    \end{equation}
\end{wzor}

\begin{wzor}{Indukcja elektryczna}
    \begin{align}
        D &= \frac{Q}{S}\\
        \Vec{D} &= \varepsilon \Vec{E}
    \end{align}
    $\Vec{E}$ -- wektor natężenia pola elektrycznego\\
    $\varepsilon$ -- przenikalność elektryczna materiału
\end{wzor}

\begin{wzor}{Prawo Gaussa}
    \begin{align}
        \Phi &= \frac{Q_{wew}}{\varepsilon_0}\\
        \Phi &= \oint\limits_S \Vec{E} \, \der \Vec{S}
    \end{align}
    $\Phi$ -- strumień pola elektrycznego\\
    $Q_{wew}$ -- ładunek elektryczny zawarty wewnątrz powierzchni $S$\\
    $S$ -- zamknięta powierzchnia, przez którą przenika ładunek (powierzchnia Gaussa)
    %%%%%%%%%%%%%%%%%%%%%%%%%%%%%%%%%%%%%%%%
    \newpage
    %%%%%%%%%%%%%%%%%%%%%%%%%%%%%%%%%%%%%%%%
    \par {\textit{Prawo Gaussa dla indukcji elektrycznej}}
    \begin{align}
        \oint\limits_S \Vec{D} \, \der\Vec{S} &= Q_{sw} \\
        \nabla \cdot \Vec{E} &= \frac{\rho_c}{\varepsilon_0}
    \end{align}
    $\Vec{D}$ -- wektor indukcji elektrycznej\\
    $Q_{sw}$ -- ładunek swobodny\\
    $\rho_c$ -- gęstość objętościowa ładunku całkowitego
\end{wzor}

\begin{wzor}{Polaryzacja dielektryka liniowego}
    \begin{align}
        \Vec{P} &= \varepsilon_0 \chi_e \Vec{E}\\
        \Vec{E} &= \Vec{E_0} - \frac{1}{\varepsilon_0} \Vec{P}
                 = \frac{1}{\chi + 1} \Vec{E_0}
                 = \frac{1}{\varepsilon} \Vec{E_0}
    \end{align}
    $\Vec{P}$ -- wektor polaryzacji dielektryka\\
    $\varepsilon_0$ -- przenikalność elektryczna w próżni\\
    $\chi_e$ -- podatność elektryczna\\
    $\Vec{E}$ -- wypadkowe pole elektryczne w dielektryku\\
    $\Vec{E_0}$ -- zewnętrzne pole elektryczne przyłożone do dielektryka\\
    $\varepsilon$ -- przenikalność względna dielektryka
\end{wzor}

\begin{wzor}{Kondensatory}
    \par\textit{Pojemność}
    \begin{align}
        C &= \frac{Q}{U}\\
        C &= \frac{S\varepsilon_0 \varepsilon_r}{d}
    \end{align}
    $S$ -- powierzchnia okładki\\
    $d$ -- odległość między okładkami
    %%%%%%%%%%%%%%%%%%%%%%%%%%%%%%%%%%%%%%%%
    \newpage
    %%%%%%%%%%%%%%%%%%%%%%%%%%%%%%%%%%%%%%%%
    \par \textit{Łączenie kondensatorów} 
    \begin{align}
        \text{Szeregowo: } \frac{1}{C_z} &= \sum\limits_{i=1}^n \frac{1}{C_i} \\
        \text{Równolegle: } C_z &= \sum\limits_{i=1}^n C_i
    \end{align}
\end{wzor}

\begin{wzor}{Prąd elektryczny}
    \begin{equation}
        I = \frac{\der q}{\der t}
    \end{equation}
    Natężeniem prądu nazywamy ładunek przepływający przez przewodnik w jednostkowym czasie
    przez przekrój tego przewodnika.
\end{wzor}

\begin{wzor}{Równanie ciągłości}
    \begin{equation}
        \nabla \cdot j = - \frac{\partial \rho}{\partial t}
    \end{equation}
    $j$ -- gęstość prądu elektrycznego\\
    $\rho$ -- gęstość ładunku elektrycznego\\
    Zmiana ładunku w pewnej objętości jest równa sumarycznemu ładunkowi,
    który wypłynął lub wpłynął przez brzeg ograniczający tę objętość.
\end{wzor}

\begin{wzor}{Siła Lorentza}
    \begin{equation}
        \Vec{F} = q(\Vec{v} \times \Vec{B})
    \end{equation}
    $\Vec{F}$ -- wektor siły\\
    $q$ -- ładunek elektryczny cząstki\\
    $\Vec{v}$ -- wektor prędkości cząstki\\
    $\Vec{B}$ -- pseudowektor indukcji magnetycznej
\end{wzor}

\begin{wzor}{Siła elektrodynamiczna}
    \begin{equation}
        \Vec{F} = I \cdot \Vec{l} \times \Vec{B}
    \end{equation}
    $I$ -- natężenie prądu płynącego przez przewodnik\\
    $\Vec{l}$ -- wektor długości przewodnika
\end{wzor}

%%%%%%%%%%%%%%%%%%%%%%%%%%%%%%%%%%%%%%%%
\newpage
%%%%%%%%%%%%%%%%%%%%%%%%%%%%%%%%%%%%%%%%

\begin{wzor}{Prawo Ohma}
    \par \textit{Prawo Ohma w postaci różniczkowej}
    \begin{equation}
        \Vec{j} = \sigma \Vec{E}
    \end{equation}
    $\Vec{j}$ -- wektor gęstości prądu w przewodniku\\
    $\sigma$ -- konduktywność (przewodnictwo właściwe)\\
    $\Vec{E}$ -- wektor pola elektrycznego w przewodniku
    \par \textit{Prawo Ohma w postaci makroskopowej}
    \begin{equation}
        U = I \cdot R
    \end{equation}
\end{wzor}

\begin{wzor}{Prawo Biota-Savarta}
    \begin{equation}
        \der\Vec{B} = \frac{\mu_0}{4\pi} \, I \, \frac{\der\Vec{l}\times \Vec{r}}{r^3} \implies
        \der B = \frac{\mu_0}{4\pi} \, I \, \frac{\der l}{r^2} \, \sin \alpha
    \end{equation}
    $\der\Vec{B}$ -- wkład do pola indukcji magnetycznej\\
    $I$ -- natężenie prądu w przewodniku\\
    $\der\Vec{l}$ -- nieskończenie mały fragment przewodnika o wektorze długości $\Vec{l}$\\
    $\Vec{r}$ -- wektor wodzący o początku w źródle pola (fragmencie przewodnika)
        i końcu w rozważanym punkcie przestrzeni\\
    $r$ -- odległość od źródła pola do rozważanego punktu przestrzeni, $r = |\Vec{r}|$\\
    $\alpha$ -- kąt pomiędzy $\der\Vec{l}$ a $\Vec{r}$
\end{wzor}

\begin{wzor}{Prawo Ampère'a}
    \begin{equation}
        \oint \Vec{B} \, \der\Vec{l} = \mu_0 I
    \end{equation}
    Prawo to wiąże indukcję magnetyczną wokół przewodnika z natężeniem prądu elektrycznego
    przepływającego w tym przewodniku.
    
    %%%%%%%%%%%%%%%%%%%%%%%%%%%%%%%%%%%%%%%%
    \newpage
    %%%%%%%%%%%%%%%%%%%%%%%%%%%%%%%%%%%%%%%%
    
    \par \textit{Prawo Ampère'a z wykorzystaniem natężenia pola magnetycznego}
    \begin{equation}
        \oint\limits_C \Vec{H} \, \der\Vec{l} = \int\limits_S \Vec{J} \, \der\Vec{a} = I
    \end{equation}
    $\Vec{H}$ -- wektor natężenia pola magnetycznego (powierzchniowego)\\
    $C$ -- krawędź ograniczająca rozpatrywaną powierzchnię\\
    $\Vec{J}$ -- gęstość prądu (powierzchniowa)\\
    $S$ -- rozpatrywana powierzchnia przewodnika ($\Vec{a}$)
    
    \par \textit{Postać różniczkowa}
    \begin{equation}
        \nabla \times \Vec{H} = \Vec{J}
    \end{equation}
\end{wzor}

\begin{wzor}{Twierdzenie Helmholtza}
    \begin{equation}
        \nabla \cdot \mathbf{A} = 0 \implies \mathbf{F} = - \nabla V + \nabla \times \mathbf{A}
    \end{equation}
    $\mathbf{F}$ -- pole wektorowe\\
    $\mathbf{A}$ -- potencjał wektorowy pola $\mathbf{F}$ o dywergencji równej 0\\
    $V$ -- potencjał skalarny pola $\mathbf{F}$
\end{wzor}

\begin{wzor}{Magnetyzacja}
    \begin{equation}
        \begin{split}
            \mathbf{M} &= \frac{\Delta \mathbf{m}}{\Delta V}\\
            \mathbf{M} &= \frac{\der \mathbf{m}}{\der V}
        \end{split}
    \end{equation}
    $\mathbf{M}$ -- pole wektorowe magnetyzacji\\
    $\mathbf{m}$ -- magnetyczny moment dipolowy\\
    $V$ -- objętość
\end{wzor}

\begin{wzor}{Prawo Faradaya}
    \begin{equation}
        \varepsilon = - \frac{\der\varphi}{\der t}
    \end{equation}
    $\varepsilon$ -- siła elektromotoryczna\\
    $\varphi$ -- strumień indukcji magnetycznej\\
    $\frac{\der\varphi}{\der t}$ -- szybkość zmian strumienia

    %%%%%%%%%%%%%%%%%%%%%%%%%%%%%%%%%%%%%%%%
    \newpage
    %%%%%%%%%%%%%%%%%%%%%%%%%%%%%%%%%%%%%%%%
    
    \par \textit{Prawo Faradaya w postaci całkowej}
    \begin{equation}
        \varepsilon = \oint\limits_C \Vec{E} \, \der\Vec{l}
            = - \frac{\der}{\der t} \int\limits_S \Vec{B} \, \der\Vec{s}
    \end{equation}
    $\Vec{E}$ -- natężenie indukowanego pola elektrycznego\\
    $C$ -- pętla z przewodnika\\
    $\Vec{B}$ -- wektor indukcji magnetycznej\\
    $S$ -- powierzchnia zamknięta pętlą $C$
    
    \par \textit{Prawo Faradaya w postaci różniczkowej}
    \begin{equation}
        \nabla \times \Vec{E} = - \frac{\partial \Vec{B}}{\partial t}
    \end{equation}
    Jest to jedno z \textbf{równań Maxwella}.
\end{wzor}

\begin{wzor}{\textbf{Równania Maxwella}}
    \par {\large \textsc{Postać całkowa}}
    \par \textit{Prawo Faradaya}
    \begin{equation}
        \begin{split}
            \oint\limits_C \Vec{E} \, \der\Vec{l} &= - \frac{\der\varphi}{\der t}\\
            \oint\limits_C \Vec{E} \, \der\Vec{l}
                &= - \frac{\der}{\der t} \int\limits_S \Vec{B} \, \der\Vec{s}
        \end{split}
    \end{equation}
    $\Vec{E}$ -- natężenie pola elektrycznego\\
    $C$ -- dowolny zamknięty kontur\\
    $\varphi$ -- strumień indukcji pola magnetycznego\\
    $\Vec{B}$ -- indukcja pola magnetycznego
    
    %%%%%%%%%%%%%%%%%%%%%%%%%%%%%%%%%%%%%%%%
    \newpage
    %%%%%%%%%%%%%%%%%%%%%%%%%%%%%%%%%%%%%%%%
    
    \par \textit{Uogólnione prawo Ampère'a}
    \begin{equation}
        \begin{split}
            \oint\limits_C \Vec{B} \, \der\Vec{l}
                &= \mu I + \mu \varepsilon \frac{\der\Phi_E}{\der t}\\
            \oint\limits_C \Vec{B} \, \der\Vec{l}
                &= \mu I + \mu \varepsilon \frac{\der}{\der t} \int\limits_S \Vec{E} \, \der\Vec{s}
        \end{split}
    \end{equation}
    $C$ -- dowolny zamknięty kontur\\
    $I$ -- całkowity prąd elektryczny przepływający przez powierzchnię $S$ wewnątrz konturu $C$\\
    $\Phi_E$ -- strumień pola elektrycznego przez tę powierzchnię\\
    $\mu$ -- przenikalność magnetyczna ośrodka\\
    $\varepsilon$ -- przenikalność elektryczna ośrodka
    
    \par \textit{Prawo Gaussa dla elektryczności}
    \begin{equation}
        \varepsilon \, \oint\limits_S \Vec{E} \, \der\Vec{s} = q
    \end{equation}
    $\varepsilon$ -- przenikalność elektryczna ośrodka\\
    $q$ -- całkowity ładunek zawarty wewnątrz powierzchni zamkniętej $S$
    
    \par \textit{Prawo Gaussa dla magnetyzmu}
    \begin{equation}
        \oint\limits_S \Vec{B} \, \der\Vec{s} = 0
    \end{equation}
    
    \par {\large \textsc{Postać różniczkowa}}
    \par \textit{Prawo Faradaya}
    \begin{equation}
        \nabla \times \Vec{E} = - \frac{\partial \Vec{B}}{\partial t}
    \end{equation}
    $\rightarrow$ Zmienne w czasie pole magnetyczne wytwarza pole elektryczne. $\leftarrow$
    
    \par \textit{Uogólnione prawo Ampère'a}
    \begin{equation}
        \nabla \times \Vec{B} = \mu \Vec{j} + \mu \varepsilon \frac{\partial \Vec{E}}{\partial t}
    \end{equation}
    $\Vec{j}$ -- gęstość prądu elektrycznego\\
    $\rightarrow$ Przepływający prąd oraz zmienne pole elektryczne wytwarzają pole magnetyczne.
    $\leftarrow$
    
    %%%%%%%%%%%%%%%%%%%%%%%%%%%%%%%%%%%%%%%%
    \newpage
    %%%%%%%%%%%%%%%%%%%%%%%%%%%%%%%%%%%%%%%%
    
    \par \textit{Prawo Gaussa dla elektryczności}
    \begin{equation}
        \varepsilon \nabla \cdot \Vec{E} = \rho
    \end{equation}
    $\rho$ -- gęstość ładunku elektrycznego\\
    $\rightarrow$ Ładunki są źródłem pola elektrycznego. $\leftarrow$
    
    \par \textit{Prawo Gaussa dla magnetyzmu}
    \begin{equation}
        \nabla \cdot \Vec{B} = 0
    \end{equation}
    $\rightarrow$ Pole magnetyczne jest bezźródłowe. $\leftarrow$
\end{wzor}

\begin{wzor}{Funkcja falowa}
    \par Funkcja falowa musi zanikać w nieskończoności, musi być ciągła oraz
    jej pochodna musi być ciągła, jeśli potencjał jest skończony.
    \par \textit{Unormowanie funkcji falowej}
    \begin{equation}
        \int\limits_{-\infty}^\infty |\Psi|^2 \, \der x = 1 \iff
            \int\limits_{-\infty}^\infty \Psi \cdot \overline{\Psi} \, \der x = 1
    \end{equation}
    \par \textit{Postulat o średniej}
    \begin{equation}
        \langle A \rangle = \int\limits_{-\infty}^\infty \overline{\Psi} \hat{A} \Psi \, \der x
    \end{equation}
    $A$ -- mierzona wielkość fizyczna, której odpowiada operator $\hat{A}$
\end{wzor}

\begin{wzor}{Operator Hamiltona}
    \begin{align}
        \hat{x_\xi} &= x_\xi\\
        \hat{p_\xi} &= -i \hbar \frac{\partial}{\partial x_\xi}\\
        \Omega(p_\xi, x_\xi) \implies \hat{\Omega} &= \Omega(\hat{p_\xi}, \hat{x_\xi})
    \end{align}
    $x_\xi$ -- współrzędna w osi $\xi$\\
    $p_\xi$ -- pęd w osi $\xi$\\
    $\Omega$ -- dowolna funkcja pędu i położenia
\end{wzor}

%%%%%%%%%%%%%%%%%%%%%%%%%%%%%%%%%%%%%%%%
\newpage
%%%%%%%%%%%%%%%%%%%%%%%%%%%%%%%%%%%%%%%%

\begin{wzor}{Równania Schrödingera}
    \par \textit{Niezależne od czasu}
    \begin{equation}
        -\frac{\hbar^2}{2m} \cdot \frac{\partial^2 \Psi}{\partial x^2} + V\Psi = E\Psi
    \end{equation}
    \par \textit{Zależne od czasu}
    \begin{equation}
        i\hbar \frac{\partial \Psi}{\partial t} =
            -\frac{\hbar^2}{2m}\cdot \frac{\partial^2 \Psi}{\partial x^2} + V\Psi
    \end{equation}
    \par \textit{Gęstość prądu prawdopodobieństwa}
    \begin{equation}
        \begin{split}
            \mathbf{j} &= \frac{i\hbar}{2m}
                (\Psi \nabla \overline{\Psi} - \overline{\Psi} \nabla \Psi)\\
            j_x &= \frac{i\hbar}{2m}
                \left( \Psi \frac{\der\overline{\Psi}}{\der x}
                    - \overline{\Psi} \frac{\der\Psi}{\der x} \right)
        \end{split}
    \end{equation}
\end{wzor}

\begin{wzor}{Model atomu Bohra}
    \begin{itemize}
        \item Elektron krąży po konkretnych stabilnych orbitach wokół jądra
            bez wypromieniowywania żadnej energii, w przeciwieństwie do tego,
            co sugeruje klasyczna elektrodynamika. Te orbity są nazywane stacjonarnymi
            i znajdują się w konkretnych odległościach od jądra atomu. Elektron
            nie może mieć żadnej orbity pomiędzy nimi.
        \item Moment pędu i promień dopuszczalnych orbit (możliwych stanów stacjonarnych)
            \begin{equation}
                L = m_e v r = n\hbar, \: n \in \mathbb{N}
            \end{equation}
            $L$ -- moment pędu elektronu\\
            $m_e$ -- masa elektronu\\
            $v$ -- prędkość elektronu\\
            $r$ -- promień orbity\\
            $n$ -- liczba naturalna ($n \in \{1, 2, 3, ...\}$)
            \begin{equation}
                \langle E_k \rangle = \frac{1}{2}nhf, \: n \in \mathbb{N}
            \end{equation}
            $\langle E_k \rangle$ -- średnia energia kinetyczna elektronu\\
            $f$ -- częstotliwość obiegu przez elektron jądra atomu
    %%%%%%%%%%%%%%%%%%%%%%%%%%%%%%%%%%%%%%%%
    \newpage
    %%%%%%%%%%%%%%%%%%%%%%%%%%%%%%%%%%%%%%%%
    
    \item Dynamiką w stanach stacjonarnych rządzi mechanika Newtonowska,
        która jednak nie obowiązuje podczas przejść między stanami stacjonarnymi.
    \item Każdej emisji lub absorbcji energii towarzyszy przejście elektronu
        między dwoma stanami stacjonarnymi.
        \begin{equation}
            \Delta E = E_2 - E_1 = h \nu
        \end{equation}
        $\nu$ -- częstotliwość fotonu
    \end{itemize}
\end{wzor}

\begin{wzor}{Entropia}
    \begin{equation}
        S = k \ln(\Delta \Gamma)
    \end{equation}
    $S$ -- entropia\\
    $k$ -- stała Boltzmanna\\
    $\Delta\Gamma$ -- liczba mikrostanów realizujących dany makrostan
\end{wzor}

\begin{wzor}{Rozkład Bosego-Einsteina}
    \begin{equation}
        \langle n_i \rangle = \frac{n}{Z} \cdot
            \frac{g_i}{\exp\big(\beta(E_i - \mu)\big) - 1}
    \end{equation}
    $\langle n_i \rangle$ -- średnia liczba cząstek w $i$-tym stanie\\
    $E_i$ -- energia $i$-tego stanu\\
    $g_i$ -- degeneracja $i$-tego stanu\\
    $n$ -- całkowita liczba cząstek\\
    $\mu$ -- potencjał chemiczny\\
    $\beta = \frac{1}{kT}$, gdzie $T$ -- temperatura w Kelvinach, $k$ -- stała Boltzmanna\\
    $Z = \sum\limits_i \frac{g_i}{\exp(\beta(E_i - \mu)) - 1}$
        -- suma statystyczna
\end{wzor}

\begin{wzor}{Rozkład Fermiego-Diraca}
    \begin{equation}
        \langle n \rangle = \frac{1}{\exp\big(\beta(E - \mu)\big) + 1}
    \end{equation}
    $\langle n \rangle$ -- średnia liczba cząstek w niezdegenerowanym stanie energetycznym $E$
\end{wzor}

%%%%%%%%%%%%%%%%%%%%%%%%%%%%%%%%%%%%%%%%
\newpage
%%%%%%%%%%%%%%%%%%%%%%%%%%%%%%%%%%%%%%%%
    
\begin{wzor}{Zliczanie stanów}
    \par \textit{W układach z bozonami}
    \begin{equation}
        W_i^B = \binom{g_i + n_i - 1}{n_i}
    \end{equation}
    $g_i$ -- liczba stanów\\
    $n_i$ -- liczba cząsteczek
    \par \textit{W układach z fermionami}
    \begin{equation}
        W_i^F = \binom{g_i}{n_i}
    \end{equation}
\end{wzor}

\begin{bottompar}
    {\footnotesize \ttfamily (CC BY-SA 4.0) 2019 Błażej Sewera - blazejok1[at]wp.pl\par
    kod źródłowy: https://github.com/jazzsewera/fog-wzory}
\end{bottompar}

%%%%%%%%%%%%%%%%%%%%%%%%%%%%%%%%%%%%%%%%
%Do not alter anything below this line.
\end{document}
